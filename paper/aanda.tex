\documentclass[longauth]{aa}  

\usepackage[switch]{lineno} % add line numbers

\usepackage{graphicx}   % Required for inserting images
\usepackage[fleqn]{amsmath}
\usepackage{array, amssymb}	% Advanced maths commands
\usepackage[utf8]{inputenc} % allow utf-8 input
\usepackage[T1]{fontenc}    % use 8-bit T1 fonts
\usepackage{url}            % simple URL typesetting
\usepackage{booktabs}       % professional-quality tables
\usepackage{lmodern}

\usepackage[colorlinks,allcolors=blue]{hyperref}


%-------------- Commands ---------------
\usepackage{xspace}
\newcommand{\gray}{$\gamma$-ray\xspace}
\newcommand{\grays}{$\gamma$-rays\xspace}

\begin{document} 

\linenumbers

  \title{gammapy v2.0 paper title}

     \subtitle{Subtitle}

   \author{A. Author\inst{1}}

   \institute{Institute one}

   \date{Received September 30, 20XX}

% maximum 300 words
  \abstract
  % context heading (optional)
  % {} leave it empty if necessary  
   {Optional, leave empty if necessary.  The heading “Context” is used when needed to
give background information on the research conducted in the paper}
  % aims heading (mandatory)
   {Mandatory. The objectives of the paper are defined here.} 
  % methods heading (mandatory)
   {Mandatory. The methods of the investigation are outlined here}
  % results heading (mandatory)
   {Mandatory. The results are summarized here.}
  % conclusions heading (optional), leave it empty if necessary
   {Optional, leave empty if necessary.  “Conclusions” can be used to
explicit the general conclusions that can be drawn from the paper.}
   
   \keywords{keywords}

   \maketitle

\section{Introduction}



\section{Conclusions}
\label{conclusions}



\begin{acknowledgements}
Acknowledgements
\end{acknowledgements}


\bibliographystyle{aa}
\bibliography{bibliography}


\begin{appendix} 




\end{appendix}


\end{document}

